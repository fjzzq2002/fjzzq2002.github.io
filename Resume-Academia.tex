\documentclass[11pt,letterpaper]{article}
\usepackage[margin=0.7in]{geometry}
\usepackage{fontspec}
\usepackage{enumitem}
\usepackage[hidelinks]{hyperref}
\usepackage{titlesec}
\usepackage{xcolor}
\usepackage{setspace}
\usepackage{comment}
\setmainfont{TeX Gyre Pagella}
%----- Colours --------------------------------------------------------------
\definecolor{navy}{HTML}{243D70}
\definecolor{linkblue}{HTML}{0645AD}
\hypersetup{colorlinks=true,linkcolor=navy,urlcolor=linkblue}

%----- Spacing --------------------------------------------------------------
\setlength{\parindent}{0pt}
\setlength{\parskip}{4pt}
\titlespacing*{\section}{0pt}{0.6\baselineskip}{0.2\baselineskip}
\titleformat{\section}{\vspace{1.5mm}\bfseries\Large\color{navy}}{}{0pt}{}

%----- Custom list environments --------------------------------------------
\newlist{outerlist}{itemize}{1}
\setlist[outerlist]{%
  label=\textbullet,
  leftmargin=1.2em,
  topsep=2pt,      % space above and below the list
  partopsep=0pt,   % extra space when environment starts a new paragraph
  itemsep=6pt,     % space between items
  parsep=0pt       % space between paragraphs within items
}

\newcommand{\daterange}[3][]{% \daterange[location]{start}{end}
  \hfill \textit{#2 -- #3} \vspace{1mm}
}

%----- Header ----------------------------------------------------------------
\newcommand{\myname}{{\LARGE \textbf{Ziqian Zhong}}}
\newcommand{\email}{\href{mailto:ziqianz@andrew.cmu.edu}{ziqianz@andrew.cmu.edu}}
\newcommand{\github}{\href{https://fjzzq2002.github.io}{fjzzq2002.github.io}}

%==========================================================================
\begin{document}
\myname\\[2pt]
{\large \email\ $\mid$\ \github}

%--------------------------------------------------------------------------
\section*{Education}
\textbf{Ph.D. in Computer Science}, Carnegie Mellon University \daterange[Pittsburgh,~PA]{01/2025}{present}\\
Conduct research around AI interpretability and safety to better understand and align AI systems. Advised by Aditi Raghunathan.

\vspace{4pt}
\textbf{B.S. in Computer Science \& Mathematics}, Massachusetts Institute of Technology \daterange[Cambridge,~MA]{08/2020}{06/2024}\\
GPA: 5.0/5.0. Selected Coursework: Quantitative Methods for NLP; Multi-agent Communication; Machine Learning; Advanced Complexity Theory; Fundamentals of Statistics.

%--------------------------------------------------------------------------
\section*{Experiences}

% TODO
\textbf{Research Scientist}, Pika Labs / Mellis Inc. \daterange[Palo Alto,~CA]{06/2024}{01/2025}\\
Developed and improved large-scale video generation models. Contributed to Pika~1.5 and led the release of Pika~2.0.

\vspace{4pt}
\textbf{Deep Learning Undergraduate Researcher}, MIT \daterange[Cambridge,~MA]{08/2022}{06/2024}\\
Worked on interpreting and understanding neural networks with Neil Thompson, Jacob Andreas, and Ziming Liu.

\vspace{4pt}
\textbf{Theoretical Computer Science Undergraduate Researcher}, MIT \daterange[Cambridge,~MA]{10/2021}{05/2022}\\
Member of research team guided by Virginia Vassilevska Williams. Co-discovered new results in graph theory and combinatorics, leading to several publications.

% \vspace{4pt}
% \textbf{Participant}, {CBAI Winter Machine Learning Bootcamp} \daterange[Cambridge,~MA]{01/2023}{02/2023}\\
% Developed skills and conducted research on mechanistic interpretability.

\vspace{4pt}
\textbf{Algo Developer Intern}, Hudson River Trading \daterange[New York,~NY]{05/2023}{08/2023}\\
Conducted market and algorithmic research; project featured in \href{https://www.hudsonrivertrading.com/hrtbeat/intern-spotlight-2023-hrt-ai-labs-summer-projects/}{intern spotlights}.

%--------------------------------------------------------------------------
\section*{Talks}
\begin{outerlist}
\item \textbf{Two Stories in Mechanistic Explanation of Neural Networks}. NeurIPS~2023 Oral with Ziming Liu.
\item \textbf{New Approach for Unbounded SubsetSum}. SODA~2023.
\end{outerlist}

%--------------------------------------------------------------------------
\section*{Publications}
\begin{outerlist}
\item \textbf{The Clock and the Pizza: Two Stories in Mechanistic Explanation of Neural Networks}. Ziqian Zhong*, Ziming Liu*, Max Tegmark, Jacob Andreas. Oral, NeurIPS~2023.
\item \textbf{Algorithmic Capabilities of Random Transformers}. Ziqian Zhong, Jacob Andreas. NeurIPS~2024.
\item \textbf{On Problems Related to Unbounded SubsetSum: A Unified Combinatorial Approach}. Mingyang Deng*, Xiao Mao*, Ziqian Zhong*. SODA~2023
\item \textbf{Grokking as Compression: A Nonlinear Complexity Perspective}. Ziming Liu, Ziqian Zhong, Max Tegmark. NeurIPS~2023 UniReps Workshop.
\item \textbf{New Additive Approximations for Shortest Paths and Cycles}. Mingyang Deng*, Yael Kirkpatrick*, Victor Rong*, Virginia Vassilevska Williams*, Ziqian Zhong*. ICALP~2022.
\item \textbf{New Lower Bounds and Upper Bounds for Listing Avoidable Vertices}. Mingyang Deng*, Virginia Vassilevska Williams*, Ziqian Zhong*. MFCS~2022.
\end{outerlist}

%--------------------------------------------------------------------------
\section*{Selected Awards}
\begin{outerlist}
\item Gold Medal, 4\textsuperscript{th} Place -- International Olympiad in Informatics 2019
\item First Place -- Meta Hackercup 2024
\item Gold Medal -- 46\textsuperscript{th} ICPC World Final
\item Honorable Mention -- Alibaba Global Mathematics Competition 2022
\item Honorable Mention -- Putnam Mathematical Competition 2022
\end{outerlist}

%--------------------------------------------------------------------------
\section*{Selected Projects}
\begin{outerlist}
\item \textbf{Is my problem new?} (\href{http://yuantiji.ac/}{Link}, 11/2023). Employ LLMs and vector embeddings to detect similarity among competitive programming problems. \sim 20k monthly page views and adopted by major contest setters.
\item \textbf{CP Ideas} (\href{https://fjzzq2002.github.io/cpideas/}{Link}, 07/2022). Generates competitive programming problems via fine-tuned GPT-3. Dataset collected and cleaned from various online judges.
\item \textbf{Understanding Transformer Sorting} (\href{https://github.com/bqi343/transformer-sorting}{Link}, 01/2023, with Benjamin Qi). Performed mechanistic interpretability to understand mechanisms in sorting transformers. Work done as part of the 2023 CBAI Winter ML Bootcamp.
\end{outerlist}

\end{document}